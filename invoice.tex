%!TEX TS-program = xelatex
%!TEX encoding = UTF-8 Unicode

\documentclass[$fontsize$, a4paper]{article}

% LAYOUT
%--------------------------------
\usepackage{geometry} 
\geometry{$geometry$, top=4.5cm, left=2.5cm, bottom=4.5cm, footskip=0cm, headsep=2cm}
% Footer
\usepackage{fancyhdr}
\pagestyle{fancy}
\fancyhf{}
\renewcommand{\headrulewidth}{0pt}


% No page numbers
\pagenumbering{gobble}

% Left align
\usepackage[document]{ragged2e}

$if(letterhead)$
  % To include the letterhead
  \usepackage{wallpaper}
  \ULCornerWallPaper{1}{letterhead.pdf}
$endif$

% TYPOGRAPHY
%--------------------------------
\usepackage{fontspec} 
\usepackage{xunicode}
\usepackage{xltxtra}
\usepackage{multirow}

% converts LaTeX specials (quotes, dashes etc.) to Unicode
\defaultfontfeatures{Mapping=tex-text}
\setromanfont [Ligatures={Common}, Numbers={OldStyle}]{$seriffont$}
\setsansfont[Scale=0.9]{$sansfont$}

% Set paragraph break
\setlength{\parskip}{1em}

% Custom ampersand
\newcommand{\amper}{{\fontspec[Scale=.95]{$seriffont$}\selectfont\itshape\&}}

$if(seriffont)$
\setmainfont[SmallCapsFeatures={LetterSpace=5,Letters=SmallCaps}]{$seriffont$}
$endif$
$if(sansfont)$
    \setsansfont{$sansfont$}
$endif$

% Command required by how Pandoc handles the list conversion
\providecommand{\tightlist}{%
  \setlength{\itemsep}{0pt}\setlength{\parskip}{0pt}}

% TABLE CUSTOMIZATION
%--------------------------------
\usepackage{tabularx}

\usepackage[compact]{titlesec} % For customizing title sections
\titlespacing*{\section}{0pt}{3pt}{-7pt} % Remove margin bottom from the title
\usepackage{arydshln} % For the dotted line on the table
\renewcommand{\arraystretch}{1.5} % Apply vertical padding to table cells
\usepackage{hhline} % For single-cell borders
\usepackage{enumitem} % For customizing lists
\setlist{nolistsep} % No whitespace around list items
\setlist[itemize]{leftmargin=0.5cm} % Reduce list left indent
\setlength{\tabcolsep}{9pt} % Larger gutter between columns


\usepackage{longtable}
\usepackage{fp}
\usepackage{booktabs}
\usepackage{numprint}

\npthousandsep{\,}

\newcounter{cnt}
\setcounter{cnt}{0}
\def\inc{\stepcounter{cnt}\thecnt}
\gdef\TotalHT{0}


% LANGUAGE
%--------------------------------
$if(lang)$
\usepackage{polyglossia}
\setmainlanguage{$polyglossia-lang.name$}
$endif$
% PDF SETUP
%--------------------------------
\usepackage[xetex, bookmarks, colorlinks, breaklinks]{hyperref}
\hypersetup
{
  pdfauthor={$author.name$},
  pdfsubject=Invoice Nr. $invoice-nr$,
  pdftitle=Invoice Nr. $invoice-nr$,
  linkcolor=blue,
  citecolor=blue,
  filecolor=black,
  urlcolor=blue
}

% To display custom date
\usepackage[nodayofweek]{datetime} 
% \newdate{date}{01}{12}{1867}
% \date{\displaydate{date}}
% Use this instead of \today: % \displaydate{date}

\def\mydate{\leavevmode\hbox{\twodigits\day.\twodigits\month.\the\year}}
\def\twodigits#1{\ifnum#1<10 0\fi\the#1}

\def\isodate{\leavevmode\hbox{\shortyear{\the\year}\twodigits\month\twodigits\day}}
\def\shortyear#1{\ifnum#1=2020 20\else \StrSubstitute{#1}{20}{} \fi}

% To split words
\usepackage{xstring}


% DOCUMENT
%--------------------------------
\begin{document}

\rhead{
  \includegraphics[width=0.35\linewidth]{FragHenri.png}
}
\small
\textsc{\textbf{$author.name$}}
$for(from)$
\textbullet{} \textsc{$from$}
$endfor$

\vspace{1em}

\normalsize \sffamily
$for(to)$
$to$\\
$endfor$

\vspace{6em}

\begin{flushright}
  \small
  $if(date.year)$
    % if a date is given
    \newdate{date}{$date.day$}{$date.month$}{$date.year$}
    \date{\displaydate{date}}
    $author.city$, \displaydate{date}
  $else$
    % otherwise use current date
    $author.city$, \today
  $endif$
\end{flushright}

\vspace{1em}

$if(invoice-nr)$
  \section*{\textsc{Rechnung} \textsc{\#$invoice-nr$}}
$else$
  $if(date.year)$
    % Short or extend every number to two digits. Add bill of the day
    \section*{\textsc{Rechnung} \textsc{\#\shortyear{$date.year$}\ifnum $date.month$<10 0\fi $date.month$\ifnum $date.day$<10 0\fi $date.day$-\ifnum $billoftheday$<10 00\else \ifnum $billoftheday$<100 0\fi\fi $billoftheday$}}
  $else$
    \section*{\textsc{Rechnung} \textsc{\#\isodate-\ifnum $billoftheday$<10 00\else \ifnum $billoftheday$<100 0\fi\fi $billoftheday$}}
  $endif$
$endif$

\footnotesize

\renewcommand\arraystretch{1.5}
\newcounter{pos}
\setcounter{pos}{0}

\setlength\dashlinedash{0.2pt}
\setlength\dashlinegap{1.5pt}
\setlength\arrayrulewidth{0.3pt}

$for(service)$
  \FPmul\temp{$service.price$}{$service.quantity$}
  \FPround\temp{\temp}{2}
  \FPadd\total{\TotalHT}{\temp}
  \FPround\total{\total}{2}
  \global\let\TotalHT\total
$endfor$

\begin{tabularx}{\linewidth}{ccXcrr}
  \hdashline[1pt/1pt]
  \noalign{\vskip 2mm}
    \textbf{Pos.} & 
    \textbf{Datum} &
    \textbf{Beschreibung} & 
    \multicolumn{1}{l}{\textbf{Menge}} &
    \multicolumn{1}{l}{\textbf{Einzelpreis}} &
    \multicolumn{1}{c}{\textbf{Gesamt}} \\
  \hline
  $for(service)$
    \noalign{\vskip 2mm}
    \inc
    & $if(service.date)$
        $service.date$
      $else$
        \mydate
      $endif$
    & $service.description$
      $if(service.details)$
        \begin{itemize}
          $for(service.details)$
            \scriptsize
            \item $service.details$
          $endfor$
        \end{itemize}
        \vspace*{-\baselineskip}\leavevmode
      $endif$
    & \numprint{$service.quantity$}
    & \FPround\two{$service.price$}{2}
      \numprint{\two}~$currency$
    & \FPmul\temp{$service.price$}{$service.quantity$}
      \FPround\temp{\temp}{2}
      \numprint{\temp}~$currency$
    \\
  $endfor$
  \noalign{\vskip 2mm}
  \hline
  \multicolumn{5}{r@{~~~}}{\textbf{Gesamtbetrag:}} & \numprint{\TotalHT}~$currency$\\

  \noalign{\vskip 2mm}
\end{tabularx}

\vspace{15mm}

\sffamily
\small
$closingnote$

\medskip

\IfFileExists{signature.pdf}
{     
      \includegraphics[height=3.5\baselineskip]{signature.pdf} \par
      $author.name$
}

\cfoot{
  \sffamily
  \begin{flushleft}
    Es wird gemäß §19 Abs. 1 Umsatzsteuergesetz keine Umsatzsteuer erhoben.\medskip\\
  \end{flushleft}
  \footnotesize
  \renewcommand{\arraystretch}{1}
  \parindent=0cm
  \begin{tabularx}{\textwidth}{
		@{}
		llll}
    \hline\\
    $author.name$ & Tel.: $author.phone$ & $bank.name$ & Steuernummer:\\
    \multirow{2}{*}{\parbox{3.5cm}{$for(from)$ $from$ \\ $endfor$}} & E-Mail: $author.email$ & Inhaber: $bank.owner$ & $author.taxnumber$\\
    & Internet: $author.website$ & IBAN: $bank.iban$ & \\
    && BIC: $bank.bic$&
  \end{tabularx}
}

\end{document}
